\section{Installation und Konfiguration}
\subsection*{Installation und Konfiguration}
\begin{frame}{Wer mag schon tippen \dots}
	\begin{itemize}
		\item \url{http://mail.cryptoparty-hd.de/}
		\item Liste der Programme für euer System
		\item Links bzw. Befehle
	\end{itemize}
\end{frame}

\blackframe

% Ablauf der Installrunde:
% Install Win:    GPG4Win(http://www.gpg4win.org/), Thunderbird
% Install MacOS:  https://gpgtools.org/ -- Apple Mail Plugin kostet Geld
% Install LinuX:  apt-get install thunderbird gnupg2 enigmail
%
% Enigmail installieren
% Enigmail:
%	Assistent ausführen (generiert 4096 RSA für 5 Jahre)
%	Schlüsselverwaltung öffnen
%		-> Ablaufdatum (was ist das)
%		-> Benutzerkennung (weitere hinzufügen, alte entfernen)
%	EMail Compose Fenster öffnen
%		-> Buttons: Meinen Schlüssel anhängen
%		-> Schickt euch keys
%	Erste Verschlüsselte EMail verfassen
%		-> Noch mal ne EMail an jemanden dessen Key ihr habt
%		-> Automatische Verschlüsselung falls bereits key bekannt

\begin{frame}{Verschlüsselte E-Mail an uns \dots}
	\begin{itemize}
		\item \url{info@michael-herbst.com}
		\item \url{jonas@letopolis.de}
	\end{itemize}
\end{frame}

\blackframe

% Enigmail weiter:
%	Mail an uns verschlüsselt mit unseren keys:
%		-> Key vom "Telefonbuch" keyserver herunterladen
%		-> Euren Key anhängen
%		-> Meldung PGP MIME (MIME verwenden)
%
% Fragen bis hierher?
%
% Facebook und 2. Key für Facebook
%      -> Key Export und Import
%
% Fragen bis hier?

\begin{frame}{Links und weitere Veranstaltungen}
	\begin{itemize}
		\item \url{http://cryptoparty-hd.de} \\[10pt]
		\item \textbf{Smartphone-Messenger} am Mi 09.12., 18:30, INF 368, Raum 432
		\item \textbf{Anonymität} am Di 12.01., 18:30, Bergheimer Str. 58, Raum 01.034
	\end{itemize}
\end{frame}

\frame{%
	\frametitle{Ende}
	\begin{center}
		\LARGE ENDE 1. Teil
	\end{center}
}


